\chapter*{Introduction}
\addcontentsline{toc}{chapter}{Introduction}

\let\thefootnote\relax\footnotetext{The research presented here was supported by the Grant Agency of Charles University in Prague (Grant-No. 272111/A-INF/MFF, principal investigator P.~{\v C}erno) and by the Czech Science Foundation (Grant-No. P103/10/0783, principal investigator I.~Mr{\'a}zov{\'a}, Grant-No. P202/10/1333, principal investigator J.~{\v S}{\'\i}ma, and Grant-No. 15-04960S, principal investigator I. Mr{\'a}zov{\'a}).}

Functional generative description (FGD) is a linguistic framework developed at Charles University in Prague since the 1960s by a team led by Petr Sgall \citep{SgNeGoHa69}. Based on the dependency grammar formalism, it is a stratificational grammar formalism that treats the sentence as a system of interlinked layers: phonological, morphematical, morphonological, analytical (surface syntax) and tectogrammatical (deep syntax). In linguistics in general and in FGD, there are various techniques used for analyzing sentences of natural languages.

Analysis by reduction \citep{LoPlKu05} is a technique used in linguistics to analyze sentences of natural languages by using a stepwise simplification of a sentence in such a way that the syntactical correctness or incorrectness of the sentence is preserved. After a finite number of steps either a correct simple sentence is obtained, or an error is detected. In this way it is also possible to determine dependencies between various parts of the given sentence, and to disambiguate between certain morphological ambiguities contained in the sentence. Restarting automaton \citep{JMPV95,O06} was invented to model the analysis by reduction. In fact, many aspects of the work on restarting automata are motivated by the basic tasks of computational linguistics. Several programs are being used in Czech and German (corpus) linguistics that are based on the idea of restarting automata.

In this thesis we study locally restricted models of restarting automata with the emphasis on the effective learnability of these models. Several restricted models have been introduced and studied intensively, starting with the so-called clearing restarting automaton \citep{CM10}, which, based on a limited context, can only delete a substring of the current content of its tape. The results obtained so far suggest that these models might be suitable for computational linguistics, since they are quite expressive and, under certain conditions, are identifiable in the limit from an informant (positive and negative samples) \citep{C12,C13}. Moreover, the instructions used in these models are human readable.

The thesis is structured as follows. In Chapter \ref{chapter:background} we provide a short survey of the theory of automata and formal languages. Chapter \ref{chapter:advanced} extends the basics of formal language theory by several selected topics, such as restarting automata, string-rewriting systems, context rewriting systems, etc., which are important with respect to our own models. Chapter \ref{chapter:goals} outlines the main goals of this thesis. The core of the thesis is split according to the outlined goals into Chapters \ref{chapter:clra}, \ref{chapter:inference} and \ref{chapter:crs_aux}. Chapter \ref{chapter:clra} is devoted to clearing restarting automata, which are the most restricted model considered in this thesis. In Chapter \ref{chapter:inference} we will see, that not using auxiliary symbols gives rise to models that are learnable under a suitable learning paradigm. Chapter \ref{chapter:crs_aux} is focused on context rewriting systems with auxiliary symbols, which are hard to learn, but the corresponding language classes are comparable to other well known language classes.

The results presented in this thesis are based on several published papers:

\begin{itemize}
\item {\v C}erno, P. and Mr{\'a}z, F.: Clearing restarting automata. In Bordinh, H., Freund, R., Holzer, M., Kutrib, M., and Otto, F., editors: \emph{Workshop on Non-Classical Models of Automata and Applications (NCMA)}. Vol. 256 of \emph{books@ocg.at}, {\"O}sterreichisches Computer Gesellschaft (2009) 77--90.
\item {\v C}erno, P. and Mr{\'a}z, F.: Clearing restarting automata. \emph{Fundamenta Informaticae}, {\bf 104}(1) (2010) 17--54.
\item {\v C}erno, P. and Mr{\'a}z, F.: $\Delta$-clearing restarting automata and $\CFL$. In Mauri, G. and Leporati, A., editors: \emph{Developments in Language Theory}. Vol. 6795 of \emph{Lecture Notes in Computer Science}, Springer Berlin / Heidelberg (2011) 153--164.
\item {\v C}erno, P. and Mr{\'a}z, F.: $\Delta$-clearing restarting automata and $\CFL$. Technical Report 8/2011, Charles University, Faculty of Mathematics and Physics, Prague (2011).
\item {\v C}erno, P.: Clearing restarting automata and grammatical inference. In Heinz, J., de la Higuera, C., and Oates, T., editors: \emph{Proceedings of the Eleventh International Conference on Grammatical Inference}. Vol. 21 of \emph{JMLR Workshop and Conference Proceedings} (2012) 54--68.
\item {\v C}erno, P.: Grammatical inference of $\lambda$-confluent context rewriting systems. In Bensch, S., Drewes, F., Freund, R., and Otto, F., editors: \emph{Workshop on Non-Classical Models of Automata and Applications (NCMA)}. Vol. 294 of \emph{books@ocg.at}, {\"O}sterreichisches Computer Gesellschaft (2013) 85--100.
\item {\v C}erno, P.: Grammatical inference of $\lambda$-confluent context rewriting systems. Technical Report 2015/1/KSVI, Charles University, Faculty of Mathematics and Physics, Prague (2015).
\item Otto, F., {\v C}erno, P., and Mr{\'a}z, F.: Limited context restarting automata and mcnaughton families of languages. In Freund, R., Holzer, M., Truthe, B., and Ultes-Nitsche, U., editors: \emph{Workshop on Non-Classical Models for Automata and Applications (NCMA)}. Vol. 290 of \emph{books@ocg.at}, {\"O}sterreichisches Computer Gesellschaft (2012) 165--180.
\item Otto, F., {\v C}erno, P., and Mr{\'a}z, F.: On the classes of languages accepted by limited context restarting automata. \emph{RAIRO - Theoretical Informatics and Applications} {\bf eFirst} (2014) 1--24. \url{http://dx.doi.org/10.1051/ita/2014001}.
\end{itemize}

These papers were cited 5 times not counting the self-citations.

%$\lambda$-confluence for context rewriting systems
%$\lambda$-Confluence Is Undecidable for Clearing Restarting Automata
%On Restarting Automata with Window Size One
%On the Determinization Blowup for Finite Automata Recognizing Equal-Length Languages
%Restarting automata for insertion languages