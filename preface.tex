\chapter*{Introduction}
\addcontentsline{toc}{chapter}{Introduction}

Functional generative description (FGD) is a linguistic framework developed at Charles University in Prague since the 1960s by a team led by Petr Sgall \cite{SgNeGoHa69}. Based on the dependency grammar formalism, it is a stratificational grammar formalism that treats the sentence as a system of interlinked layers: phonological, morphematical, morphonological, analytical (surface syntax) and tectogrammatical (deep syntax).

Analysis by reduction \cite{LoPlKu05} is a technique used in linguistics to analyze sentences of natural languages by using a stepwise simplification of a sentence in such a way that the syntactical correctness or incorrectness of the sentence is preserved. After a finite number of steps either a correct simple sentence is obtained, or an error is detected. In this way it is also possible to determine dependencies between various parts of the given sentence, and to disambiguate between certain morphological ambiguities contained in the sentence. Restarting automaton \cite{JMPV95,O06} was invented to model the analysis by reduction. In fact, many aspects of the work on restarting automata are motivated by the basic tasks of computational linguistics. Several programs are being used in Czech and German (corpus) linguistics that are based on the idea of restarting automata. 

The main focus of this thesis is the study of the locally restricted models of restarting automata with the emphasis on the effective learnability of these models under suitable learning paradigms. Several restricted models have been introduced and studied intensively, starting with the most restricted model, the so-called clearing restarting automaton \cite{CM10}, which, based on a limited context, can only delete a substring of the current content of its tape. The results obtained so far suggest that these models might be suitable for computational linguistics, since they are quite expressive and, under certain conditions, are identifiable in the limit from an informant (positive and negative samples) \cite{C12}. Moreover, the instructions used in these models are human readable.