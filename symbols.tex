\chapwithtoc{List of Symbols}

\noindent{\bf Standard mathematical symbols}.

\begin{enumerate}[]
\item $\N$: set of (positive) natural numbers.
\end{enumerate}

\noindent{\bf Words and languages} (see Section \ref{section:words-and-languages}).

\begin{enumerate}[]
\item $a, b, c, d$: symbols usually reserved for letters.
\item $u, v, w, x, y, z$: symbols usually reserved for words.
\item $\Sigma$: usually denotes an (input) alphabet.
\item $\Gamma$: usually denotes a working alphabet (with auxiliary symbols).
\item $L, L_0, L_1, L_2$: symbols usually reserved for languages (over $\Sigma$ or $\Gamma$).
\item $\Sigma^*$: set of all words over $\Sigma$.
\item $\Sigma^+$: set of all nonempty words over $\Sigma$.
\item $\Sigma^k$: set of all words over $\Sigma$ of length $k$.
\item $\Sigma^{\le k}$: set of all words over $\Sigma$ of length at most $k$.
\item $\lambda$: the empty word.
\item $|w|$: length of the word $w$.
\item $|w|_a$: number of occurences of the letter $a$ in the word $w$.
\item $u \cdot v$ (or $uv$): catenation (or concatenation) of the words $u$ and $v$.
\item $w^i$: is defined as follows: $w^0 = \lambda$, $w^{i+1} = w^i \cdot w$.
\item $w^R$: reversal (mirror image) of a word $w$.
\item $\Pref_k(u)$: prefix of length $k$ of the word $u$.
\item $\Suff_k(u)$: suffix of length $k$ of the word $u$.
\item $\Int_k(u)$: interior words of length $k$ that occur in $u$.
\item $L_1 \cup L_2$: union of languages $L_1$ and $L_2$.
\item $L_1 \cap L_2$: intersection of languages $L_1$ and $L_2$.
\item $L_1 \cdot L_2$ (or $L_1 L_2$): catenation (or concatenation) of languages $L_1$ and $L_2$.
\item $L^i$: is defined as follows: $L^0 = \{\lambda\}$, $L^{i+1} = L^i \cdot L$.
\item $L^*$ ($L^+$, respectively): Kleene star (Kleene plus, respectively) of a language $L$.
\item $L^R$: reversal (mirror image) of a language $L$.
\item $\delta_x(L) = x \backslash L$: left-derivate of $L$ with respect to $x \in \Sigma^*$.
\item $\delta^x(L) = L / x$: right-derivate of $L$ with respect to $x \in \Sigma^*$.
\item $L_0 \backslash L$: left-quotient of $L$ by $L_0$.
\item $L / L_0$: right-quotient of $L$ by $L_0$.
\item $2^S$ (or \index{$\mathcal{P}(S)$}$\mathcal{P}(S)$): power set of $S$, i.\,e.\ the collection of all subsets of $S$.
\item $\calLL$: set of languages.
\item $\calM$: set of models (automata, grammars, context rewriting systems, etc.).
\item $\calL{\calM}$: set of languages accepted (recognized, generated, etc.) by the set of models (automata, grammars, context rewriting systems, etc.) $\calM$.
\item $\calA$: symbol denoting an algorithm.
\item $\Fin$: finite languages.
\end{enumerate}

\noindent{\bf Regular Languages} (see Section \ref{section:regular-languages}).

\begin{enumerate}[]
\item $\Reg$: regular languages.
\end{enumerate}

\noindent{\em Finite Automata} (see Section \ref{subsection:finite-automata}).

\begin{enumerate}[]
\item $A$: usually denotes a finite automaton.
\item $Q$: usually denotes a finite set of states.
\item $\FA$: finite automaton.
\item $\DFA$: deterministic finite automaton.
\item $\NFA$: nondeterministic finite automaton.
\item $\lambda$-$\NFA$: nondeterministic finite automaton with $\lambda$-transitions.
\item $\NNFA$: $\NFA$ with nondeterministic starting state.
\end{enumerate}

\noindent{\bf Context-free languages} (see Section \ref{section:context-free-languages}).

\begin{enumerate}[]
\item $\CFL$: context-free languages.
\item $V_T$: usually denotes a set of terminals (or terminal letters).
\item $V_N$: usually denotes a set of nonterminals (or variables).
\item $G$: usually denotes a grammar.
\item $L(G)$ or $L_G$: usually a language generated by the grammar $G$.
\end{enumerate}

\noindent{\em Pushdown machines} (see Section \ref{subsection:pushdown-machines}).

\begin{enumerate}[]
\item $\pdm$: pushdown machine.
\item $\pda$: pushdown automaton.
\item $\dpda$: deterministic pushdown automaton.
\item $\DCFL$: deterministic context-free languages.
\item $\symDCFL$: $\symDCFL = \DCFL\cap\DCFL^R$.
\end{enumerate}

% TODO(petercerno): Define!
%$\TPDA$

\noindent{\em Subfamilies} (see Section \ref{subsection:context-free-subfamilies}).

\begin{enumerate}[]
\item $\Lin$: linear languages.
\item $\Ocl$: one-counter languages.
\item $\LL(k)$-language: language generated by a $\LL(k)$-grammar.
\item $\LR(k)$-language: language generated by a $\LR(k)$-grammar.
\end{enumerate}

\noindent{\bf Chomsky Hierarchy} (see Section \ref{section:chomsky-hierarchy}).

\begin{enumerate}[]
\item $\RE$ ($\mathcal{L}_0$): recursively enumerable languages.
\item $\CSL$ ($\mathcal{L}_1$): context-sensitive languages.
\item $\CFL$ ($\mathcal{L}_2$): context-free languages.
\item $\Reg$ ($\mathcal{L}_4$): regular languages.
\end{enumerate}

\noindent{\em Context-Sensitive Grammars} (see Section \ref{subsection:context-sensitive-grammars}).

\begin{enumerate}[]
\item $\LBA$: linear bounded automaton.
\item $\DCSL$: deterministic context-sensitive languages.
\item $\GCSL$: growing context-sensitive languages.
\end{enumerate}

\noindent{\bf Computational complexity theory}.

\begin{enumerate}[]
\item $\SAT$: boolean satisfiability problem.
\item $\classP$: decision problems decidable in deterministic polynomial time.
\item $\classNP$: decision problems decidable in nondeterministic polynomial time.
\item $\classDSPACE(n)$: decision problems decidable in deterministic linear space.
\item $\classNSPACE(n)$: decision problems decidable in nondeterministic linear space.
\item $\classPSPACE$: decision problems decidable in polynomial space.
\end{enumerate}

\noindent{\bf Restarting automata} (see Section \ref{section:restarting-automata}).

\begin{enumerate}[]
\item $\cent$: left sentinel.
\item $\$$: right sentinel.
\item $\MVR$: move-right step causes a restarting automaton to shift the read/write window one position to the right and to enter a new state.
\item $\MVL$: move-left step causes a restarting automaton to shift the read/write window one position to the left and to enter a new state.
\item $\Rewrite$: rewrite step causes a restarting automaton to replace the content of the read/write window by a new shorter string, thereby shortening the tape, and to enter state a new state. Further, the read/write window is placed immediately to the right of the new string.
\item $\Restart$: restart step causes a restarting automaton to place its read/write window over the left end of the tape, so that the first symbol it sees is the left border marker $\cent$, and to reenter the initial state $q_0$.
\item $\Accept$: accept step causes a restarting automaton to halt and accept.
\item $\RLWW$-automaton: the most general restarting automaton. Each cycle of each computation of an $\RLWW$-automaton $M$ consists of three phases: first $M$ scans its tape performing $\MVR$- and $\MVL$-instructions, then it executes a $\Rewrite$ step, and finally it scans its tape again performing $\MVR$- and $\MVL$-instructions.
\item $\RLW$-automaton: $\RLWW$-automaton such that its tape alphabet coincides with its input alphabet $\Sigma$, that is, no auxiliary symbols are available.
\item $\RL$-automaton: $\RLW$-automaton for which the right-hand side of each $\Rewrite$ step is a scattered subword of the left-hand side.
\item $\RRWW$-automaton: $\RLWW$-automaton that does not use any $\MVL$-steps. Thus, in each cycle an $\RRWW$-automaton can scan its tape only once from left to right.
\item $\RRW$-automaton: $\RRWW$-automaton such that its tape alphabet coincides with its input alphabet $\Sigma$, that is, no auxiliary symbols are available.
\item $\RR$-automaton: $\RRW$-automaton for which the right-hand side of each $\Rewrite$ step is a scattered subword of the left-hand side.
\item $\RWW$-automaton: $\RRWW$-automaton such that it makes restart immediately after performing a $\Rewrite$ operation. In particular, this means that it cannot perform a rewrite step during the tail of a computation.
\item $\RW$-automaton: $\RWW$-automaton such that its tape alphabet coincides with its input alphabet $\Sigma$, that is, no auxiliary symbols are available.
\item $\R$-automaton: $\RW$-automaton for which the right-hand side of each $\Rewrite$ step is a scattered subword of the left-hand side.
\item $\monPrefix$: The prefix $\monPrefix$ denotes the classes of monotone restarting automata.
\item $\detPrefix$: The prefix $\detPrefix$ denotes deterministic classes of restarting automata.
\end{enumerate}

\noindent{\bf String-rewriting systems} (see Section \ref{section:string-rewriting-systems}).

\begin{enumerate}[]
\item $R$: usually denotes a string-rewriting system.
\item $\SRS$: string-rewriting system.
\item $\dom(R)$: domain of a $\SRS$ $R$.
\item $\rng(R)$: range of a $\SRS$ $R$.
\item $\|R\|$: size of a $\SRS$ $R$.
\item $u \to_R v$: single-step reduction relation induced by a $\SRS$ $R$.
\item $u \to_R^* v$: reflexive and transitive closure of $\to_R$.
\item $\Delta_R^*(u)$: set of all descendants of $u$ (modulo a $\SRS$ $R$).
\item $\nabla_R^*(v)$: set of all ancestors of $v$ (modulo a $\SRS$ $R$).
\item $\IRR(R)$: set of all irreducible strings (modulo a $\SRS$ $R$).
\item $\CRL$: Church-Rosser languages.
\end{enumerate}

\noindent{\em McNaughton Families of Languages} (see Section \ref{subsection:mcnaughton-families}).

\begin{enumerate}[]
\item $\McNL$: McNaughton languages (defined by finite string-rewriting systems).
\item $\wrMcNL$: McNaughton languages defined by finite weight-reducing string-rewriting systems.
\item $\lrMcNL$: McNaughton languages defined by finite length-reducing string-rewriting systems.
\item $\monMcNL$: McNaughton languages defined by finite monadic string-rewriting systems.
\item $\genmonMcNL$: McNaughton languages defined by finite generalized monadic string-rewriting systems.
\item $\conMcNL$: McNaughton languages defined by finite confluent string-rewriting systems.
\item $\conwrMcNL$: McNaughton languages defined by finite confluent weight-reducing string-rewriting systems.
\item $\conlrMcNL$: McNaughton languages defined by finite confluent length-reducing string-rewriting systems.
\item $\conmonMcNL$: McNaughton languages defined by finite confluent monadic string-rewriting systems.
\item $\congenmonMcNL$: McNaughton languages defined by finite confluent generalized monadic string-rewriting systems.
\end{enumerate}

\noindent{\em Delimited String-Rewriting Systems} (see Section \ref{section:delimited-string-rewriting-systems}).

\begin{enumerate}[]
\item $\DSRS$ delimited string-rewriting system.
\item $\Apply_{\mathcal{R}}(w)$: string obtained by applying different rules in $\DSRS$ $\mathcal{R}$ until no more rules can be applied.
\item $\NewRule$: generates the next possible rule to be checked.
\item $\Consistent$: checks that by adding a new rule to the system, one does not rewrite a positive example and a negative example into the same string.
\end{enumerate}

\noindent{\bf Context rewriting systems and derived models} (see Section \ref{section:context-rewriting-systems} and Definitions \ref{definition:restrictions} and \ref{definition:derived-classes}).

\begin{enumerate}[]
\item $\CRS$: context rewriting system.
\item $\kCRS$: context rewriting system with length of contexts equal to $k$.
\item $M$: usually denotes a context rewriting system.
\item $\calM$: usually denotes a set (class) of context rewriting systems.
\item $LC_{\cdot}$, $LC_k$: set of left contexts.
\item $RC_{\cdot}$, $RC_k$: set of right contexts.
\item $|M|$: width of the $\CRS$ $M$.
\item $\size(M)$: size of the $\CRS$ $M$.
\item $M_{\sf left}$: $\CRS$ $M$ with left-most rewriting relation $\vdash^{\sf left}_M$.
\item $\kcalM$: set of all $\CRS$ $M \in \calM$ such that every instruction $\phi$ of $M$ has length of contexts equal to $k$.
\item $\llcalM$: set of all $\CRS$ $M \in \calM$ such that every instruction $\phi$ of $M$ has width at most $l$.
\item $\klcalM$: set of all $\CRS$ $M \in \kcalM$ such that every instruction $\phi$ of $M$ has width at most $l$.
\item $\concalM$: set of all $\CRS$s $M \in \calM$ such that $M$ is confluent.
\item $\lconcalM$: set of all $\CRS$s $M \in \calM$ such that $M$ is $\lambda$-confluent.
\item $\leftcalM$: set $\{M_{\sf left} \mid M \in \calM\}$.
\item $\clRA$: clearing restarting automaton.
\item $\kclRA$: clearing restarting automaton with length of contexts equal to $k$.
\item $\zlclRA$: clearing restarting automaton with empty contexts and maximal width of instructions equal to $l$.
\item $\klclRA$: clearing restarting automaton with length of contexts equal to $k$ and maximal width of instructions equal to $l$.
\item $\sclRA$: subword-clearing restarting automaton.
\item $\DclRA$: $\Delta$-clearing restarting automaton.
\item $\kDclRA$: $\Delta$-clearing restarting automaton with length of contexts equal to $k$.
\item $\DXclRA$: $\Delta^*$-clearing restarting automaton.
\item $\kDXclRA$: $\Delta^*$-clearing restarting automaton with length of contexts equal to $k$. 
\end{enumerate}

\noindent{\bf Other Models} (see Section \ref{section:other-models}).

\noindent{\em Marcus Contextual Grammars} (see Section \ref{section:marcus-contextual-grammars}).

\begin{enumerate}[]
\item $\TC$: languages generated by total contextual grammars.
\item $\ECC$: languages externally generated by contextual grammars.
\item $\ICC$: languages internally generated by contextual grammars.
\item $\EC$: languages externally generated by contextual grammars without choice.
\item $\IC$ languages internally generated by contextual grammars without choice.
\end{enumerate}

\noindent{\em Pure Grammars} (see Section \ref{section:pure-grammars}).

\begin{enumerate}[]
\item $\PCF$: pure context-free grammars (languages).
\item $\PLI$: pure length increasing grammars (languages).
\end{enumerate}

\noindent{\bf $\Delta$-Clearing Restarting Automata} (see Section \ref{section:dxclra}).

\noindent{\em Algorithmic Viewpoint} (see Section \ref{section:dxclra_viewpoint}).

\begin{enumerate}[]
\item $(Q, S)$: protocol consisting of a querier $Q$ and a solver $S$.
\item $L(Q, S)$ language recognized by the protocol $(Q, S)$.
\item $\calQ$: class of queriers.
\item $\calS$: class of solvers.
\item $\calL{\calQ, \calS}$: class of languages recognized by the queriers $\calQ$ and solvers $\calS$.
\end{enumerate}

\noindent{\bf Limited Context Restarting Automata} (see Section \ref{section:lcra}).

\begin{enumerate}[]
\item $\ACSL$: acyclic context-sensitive languages.
\item $\GACSL$: growing acyclic context-sensitive languages.
\item $\lcRA$: limited context restarting automaton.
\item $\lcRA[i]$: $\lcRA$ of type~$\mathcal{R}_i$.
\item $\lcRAp[i]$: $\lcRA$ of type~$\mathcal{R}_i'$.
\item $\conlcRA[i]$: confluent $\lcRA$ of type~$\mathcal{R}_i$.
\item $\conlcRAp[i]$: confluent $\lcRA$ of type~$\mathcal{R}_i'$.
\end{enumerate}
